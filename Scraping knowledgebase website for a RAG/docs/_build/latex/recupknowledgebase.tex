%% Generated by Sphinx.
\def\sphinxdocclass{report}
\documentclass[letterpaper,10pt,english]{sphinxmanual}
\ifdefined\pdfpxdimen
   \let\sphinxpxdimen\pdfpxdimen\else\newdimen\sphinxpxdimen
\fi \sphinxpxdimen=.75bp\relax
\ifdefined\pdfimageresolution
    \pdfimageresolution= \numexpr \dimexpr1in\relax/\sphinxpxdimen\relax
\fi
%% let collapsible pdf bookmarks panel have high depth per default
\PassOptionsToPackage{bookmarksdepth=5}{hyperref}

\PassOptionsToPackage{booktabs}{sphinx}
\PassOptionsToPackage{colorrows}{sphinx}

\PassOptionsToPackage{warn}{textcomp}
\usepackage[utf8]{inputenc}
\ifdefined\DeclareUnicodeCharacter
% support both utf8 and utf8x syntaxes
  \ifdefined\DeclareUnicodeCharacterAsOptional
    \def\sphinxDUC#1{\DeclareUnicodeCharacter{"#1}}
  \else
    \let\sphinxDUC\DeclareUnicodeCharacter
  \fi
  \sphinxDUC{00A0}{\nobreakspace}
  \sphinxDUC{2500}{\sphinxunichar{2500}}
  \sphinxDUC{2502}{\sphinxunichar{2502}}
  \sphinxDUC{2514}{\sphinxunichar{2514}}
  \sphinxDUC{251C}{\sphinxunichar{251C}}
  \sphinxDUC{2572}{\textbackslash}
\fi
\usepackage{cmap}
\usepackage[T1]{fontenc}
\usepackage{amsmath,amssymb,amstext}
\usepackage{babel}



\usepackage{tgtermes}
\usepackage{tgheros}
\renewcommand{\ttdefault}{txtt}



\usepackage[Bjarne]{fncychap}
\usepackage{sphinx}

\fvset{fontsize=auto}
\usepackage{geometry}


% Include hyperref last.
\usepackage{hyperref}
% Fix anchor placement for figures with captions.
\usepackage{hypcap}% it must be loaded after hyperref.
% Set up styles of URL: it should be placed after hyperref.
\urlstyle{same}

\addto\captionsenglish{\renewcommand{\contentsname}{Contents:}}

\usepackage{sphinxmessages}
\setcounter{tocdepth}{1}



\title{RecupKnowledgeBase}
\date{Jul 23, 2024}
\release{}
\author{Maxime Bouet}
\newcommand{\sphinxlogo}{\vbox{}}
\renewcommand{\releasename}{}
\makeindex
\begin{document}

\ifdefined\shorthandoff
  \ifnum\catcode`\=\string=\active\shorthandoff{=}\fi
  \ifnum\catcode`\"=\active\shorthandoff{"}\fi
\fi

\pagestyle{empty}
\sphinxmaketitle
\pagestyle{plain}
\sphinxtableofcontents
\pagestyle{normal}
\phantomsection\label{\detokenize{index::doc}}


\sphinxstepscope


\chapter{RecupKnowledgeBase}
\label{\detokenize{modules:recupknowledgebase}}\label{\detokenize{modules::doc}}
\sphinxstepscope


\section{Project package}
\label{\detokenize{Project:project-package}}\label{\detokenize{Project::doc}}

\subsection{Subpackages}
\label{\detokenize{Project:subpackages}}
\sphinxstepscope


\subsubsection{Project.Script package}
\label{\detokenize{Project.Script:project-script-package}}\label{\detokenize{Project.Script::doc}}

\paragraph{Submodules}
\label{\detokenize{Project.Script:submodules}}

\paragraph{Project.Script.RecupKbFortinet module}
\label{\detokenize{Project.Script:module-Project.Script.RecupKbFortinet}}\label{\detokenize{Project.Script:project-script-recupkbfortinet-module}}\index{module@\spxentry{module}!Project.Script.RecupKbFortinet@\spxentry{Project.Script.RecupKbFortinet}}\index{Project.Script.RecupKbFortinet@\spxentry{Project.Script.RecupKbFortinet}!module@\spxentry{module}}
\sphinxAtStartPar
This script retrieves the content of Fortinet’s knowledge base and saves it to a file.
\index{curlAllArticles() (in module Project.Script.RecupKbFortinet)@\spxentry{curlAllArticles()}\spxextra{in module Project.Script.RecupKbFortinet}}

\begin{fulllineitems}
\phantomsection\label{\detokenize{Project.Script:Project.Script.RecupKbFortinet.curlAllArticles}}
\pysigstartsignatures
\pysiglinewithargsret{\sphinxcode{\sphinxupquote{Project.Script.RecupKbFortinet.}}\sphinxbfcode{\sphinxupquote{curlAllArticles}}}{}{}
\pysigstopsignatures
\sphinxAtStartPar
Download the content of each URL from ‘Temp/SecondURLs.txt’ and saves it to a folder named after the article title.

\sphinxAtStartPar
Parameters:
None
\begin{description}
\sphinxlineitem{Returns:}
\sphinxAtStartPar
None

\end{description}

\end{fulllineitems}

\index{generateAllArticleUrls() (in module Project.Script.RecupKbFortinet)@\spxentry{generateAllArticleUrls()}\spxextra{in module Project.Script.RecupKbFortinet}}

\begin{fulllineitems}
\phantomsection\label{\detokenize{Project.Script:Project.Script.RecupKbFortinet.generateAllArticleUrls}}
\pysigstartsignatures
\pysiglinewithargsret{\sphinxcode{\sphinxupquote{Project.Script.RecupKbFortinet.}}\sphinxbfcode{\sphinxupquote{generateAllArticleUrls}}}{}{}
\pysigstopsignatures
\sphinxAtStartPar
Generate all the article URLs for the Fortinet community website.

\sphinxAtStartPar
This function reads the URLs from the ‘Temp/FirstURLs.txt’ file, retrieves the page source for each URL using the ‘selenium’ function,
and extracts the article URLs from the page source. The extracted URLs are saved in the ‘FirstUrls’ list.

\sphinxAtStartPar
The function then removes any duplicate URLs from the ‘FirstUrls’ list and saves the remaining URLs in the ‘Temp/SecondURLs.txt’ file,
each URL followed by a newline character.
\begin{description}
\sphinxlineitem{Parameters:}
\sphinxAtStartPar
None

\sphinxlineitem{Returns:}
\sphinxAtStartPar
None

\end{description}

\end{fulllineitems}

\index{generateAllPagesUrls() (in module Project.Script.RecupKbFortinet)@\spxentry{generateAllPagesUrls()}\spxextra{in module Project.Script.RecupKbFortinet}}

\begin{fulllineitems}
\phantomsection\label{\detokenize{Project.Script:Project.Script.RecupKbFortinet.generateAllPagesUrls}}
\pysigstartsignatures
\pysiglinewithargsret{\sphinxcode{\sphinxupquote{Project.Script.RecupKbFortinet.}}\sphinxbfcode{\sphinxupquote{generateAllPagesUrls}}}{\sphinxparam{\DUrole{n}{url}}}{}
\pysigstopsignatures
\sphinxAtStartPar
Generate all the URLs for the pages related to a specific knowledge base on the Fortinet community website.
\begin{description}
\sphinxlineitem{Args:}
\sphinxAtStartPar
url (str): The URL of the main page of the knowledge base.

\sphinxlineitem{Returns:}
\sphinxAtStartPar
None

\end{description}

\sphinxAtStartPar
This function first saves the HTML content of the main page of the knowledge base to a file named ‘Temp/knowledge\_base.html’.
It then reads the content of the file and extracts the URLs of the individual pages using regular expressions.
The extracted URLs are stored in the list ‘FirstUrls’.

\sphinxAtStartPar
Next, the function iterates over each URL in ‘FirstUrls’ and retrieves the page source using the ‘selenium’ function.
The page source is saved to a file named ‘Temp/knowledge.html’.
The function then extracts the maximum number of pages from the page source using regular expressions and stores it in ‘max\_pages’.
If ‘max\_pages’ is greater than 0, the function generates additional URLs for each page number from 2 to ‘max\_pages’ and stores them in the list ‘tempsURLs’.

\sphinxAtStartPar
Finally, the function appends all the URLs in ‘FirstUrls’ and ‘tempsURLs’ to a file named ‘Temp/FirstURLs.txt’.

\end{fulllineitems}

\index{main() (in module Project.Script.RecupKbFortinet)@\spxentry{main()}\spxextra{in module Project.Script.RecupKbFortinet}}

\begin{fulllineitems}
\phantomsection\label{\detokenize{Project.Script:Project.Script.RecupKbFortinet.main}}
\pysigstartsignatures
\pysiglinewithargsret{\sphinxcode{\sphinxupquote{Project.Script.RecupKbFortinet.}}\sphinxbfcode{\sphinxupquote{main}}}{}{}
\pysigstopsignatures
\sphinxAtStartPar
Execute the entire process of generating all pages URLs, generating all article URLs, and scraping all articles for the Fortinet community website.

\sphinxAtStartPar
This function performs the following steps:
1. Sets the initial URL to ‘\sphinxurl{https://community.fortinet.com/t5/Knowledge-Base/ct-p/knowledgebase}’.
2. Calls the ‘generateAllPagesUrls’ function to generate all the pages URLs.
3. Prints ‘Getting all URLs …’ to indicate the process of generating URLs.
4. Calls the ‘generateAllArticleUrls’ function to generate all the article URLs.
5. Prints ‘Scraping all URLs …’ to indicate the process of scraping URLs.
6. Calls the ‘curlAllArticles’ function to scrape all the articles.
7. Removes all temporary files ‘Temp/knowledge.html’, ‘Temp/FirstURLs.txt’, Temp/SecondURLs.txt’, and ‘Temp/knowledge\_base.html’.

\sphinxAtStartPar
This function does not take any parameters and does not return any values.

\end{fulllineitems}

\index{save\_knowledge\_base() (in module Project.Script.RecupKbFortinet)@\spxentry{save\_knowledge\_base()}\spxextra{in module Project.Script.RecupKbFortinet}}

\begin{fulllineitems}
\phantomsection\label{\detokenize{Project.Script:Project.Script.RecupKbFortinet.save_knowledge_base}}
\pysigstartsignatures
\pysiglinewithargsret{\sphinxcode{\sphinxupquote{Project.Script.RecupKbFortinet.}}\sphinxbfcode{\sphinxupquote{save\_knowledge\_base}}}{\sphinxparam{\DUrole{n}{url}\DUrole{p}{:}\DUrole{w}{ }\DUrole{n}{str}}\sphinxparamcomma \sphinxparam{\DUrole{n}{file\_name}\DUrole{p}{:}\DUrole{w}{ }\DUrole{n}{str}}}{{ $\rightarrow$ None}}
\pysigstopsignatures
\sphinxAtStartPar
Save the content of a webpage to a file.
\begin{description}
\sphinxlineitem{Args:}
\sphinxAtStartPar
url (str): The URL of the webpage to save.
file\_name (str): The name of the file to save the content to.

\sphinxlineitem{Returns:}
\sphinxAtStartPar
None

\sphinxlineitem{Raises:}
\sphinxAtStartPar
None

\end{description}

\sphinxAtStartPar
This function sends a GET request to the specified URL and retrieves the HTML content of the webpage.
It then uses BeautifulSoup to parse the HTML and prettify it. The prettified HTML is then written to the specified file.
\begin{description}
\sphinxlineitem{Example usage:}
\sphinxAtStartPar
save\_knowledge\_base(’\sphinxurl{https://www.example.com}’, ‘knowledge\_base.html’)

\end{description}

\end{fulllineitems}

\index{selenium() (in module Project.Script.RecupKbFortinet)@\spxentry{selenium()}\spxextra{in module Project.Script.RecupKbFortinet}}

\begin{fulllineitems}
\phantomsection\label{\detokenize{Project.Script:Project.Script.RecupKbFortinet.selenium}}
\pysigstartsignatures
\pysiglinewithargsret{\sphinxcode{\sphinxupquote{Project.Script.RecupKbFortinet.}}\sphinxbfcode{\sphinxupquote{selenium}}}{\sphinxparam{\DUrole{n}{url}}}{}
\pysigstopsignatures
\sphinxAtStartPar
Open a Firefox browser in headless mode and retrieves the page source of the specified URL.
\begin{description}
\sphinxlineitem{Args:}
\sphinxAtStartPar
url (str): The URL of the webpage to load.

\sphinxlineitem{Returns:}
\sphinxAtStartPar
str: The page source of the loaded webpage.

\end{description}

\end{fulllineitems}



\paragraph{Project.Script.RecupKbNETSKOPE module}
\label{\detokenize{Project.Script:module-Project.Script.RecupKbNETSKOPE}}\label{\detokenize{Project.Script:project-script-recupkbnetskope-module}}\index{module@\spxentry{module}!Project.Script.RecupKbNETSKOPE@\spxentry{Project.Script.RecupKbNETSKOPE}}\index{Project.Script.RecupKbNETSKOPE@\spxentry{Project.Script.RecupKbNETSKOPE}!module@\spxentry{module}}
\sphinxAtStartPar
This module retrieves the content of NetSkope’s knowledge base and saves it to a file.
\index{extract\_urls() (in module Project.Script.RecupKbNETSKOPE)@\spxentry{extract\_urls()}\spxextra{in module Project.Script.RecupKbNETSKOPE}}

\begin{fulllineitems}
\phantomsection\label{\detokenize{Project.Script:Project.Script.RecupKbNETSKOPE.extract_urls}}
\pysigstartsignatures
\pysiglinewithargsret{\sphinxcode{\sphinxupquote{Project.Script.RecupKbNETSKOPE.}}\sphinxbfcode{\sphinxupquote{extract\_urls}}}{\sphinxparam{\DUrole{n}{element}}}{}
\pysigstopsignatures
\sphinxAtStartPar
Extract the URL from an HTML element if it exists.
\begin{description}
\sphinxlineitem{Args:}
\sphinxAtStartPar
element (bs4.element.Tag): The HTML element to extract the URL from.

\sphinxlineitem{Returns:}
\sphinxAtStartPar
str or None: The extracted URL if it exists, None otherwise.

\end{description}

\end{fulllineitems}

\index{generateAllPagesUrls() (in module Project.Script.RecupKbNETSKOPE)@\spxentry{generateAllPagesUrls()}\spxextra{in module Project.Script.RecupKbNETSKOPE}}

\begin{fulllineitems}
\phantomsection\label{\detokenize{Project.Script:Project.Script.RecupKbNETSKOPE.generateAllPagesUrls}}
\pysigstartsignatures
\pysiglinewithargsret{\sphinxcode{\sphinxupquote{Project.Script.RecupKbNETSKOPE.}}\sphinxbfcode{\sphinxupquote{generateAllPagesUrls}}}{\sphinxparam{\DUrole{n}{file}}}{}
\pysigstopsignatures
\sphinxAtStartPar
Extract URLs from the given HTML file and writes them to an output file.
\begin{description}
\sphinxlineitem{Args:}
\sphinxAtStartPar
file (str): The path to the HTML file.

\sphinxlineitem{Returns:}
\sphinxAtStartPar
None

\end{description}

\end{fulllineitems}

\index{getContent() (in module Project.Script.RecupKbNETSKOPE)@\spxentry{getContent()}\spxextra{in module Project.Script.RecupKbNETSKOPE}}

\begin{fulllineitems}
\phantomsection\label{\detokenize{Project.Script:Project.Script.RecupKbNETSKOPE.getContent}}
\pysigstartsignatures
\pysiglinewithargsret{\sphinxcode{\sphinxupquote{Project.Script.RecupKbNETSKOPE.}}\sphinxbfcode{\sphinxupquote{getContent}}}{\sphinxparam{\DUrole{n}{file\_path}\DUrole{p}{:}\DUrole{w}{ }\DUrole{n}{str}}}{{ $\rightarrow$ None}}
\pysigstopsignatures
\sphinxAtStartPar
Read URLs from the given file and saves their content to the local file system.
\begin{description}
\sphinxlineitem{Args:}
\sphinxAtStartPar
file\_path (str): The path to the file containing the URLs.

\sphinxlineitem{Returns:}
\sphinxAtStartPar
None

\end{description}

\end{fulllineitems}

\index{main() (in module Project.Script.RecupKbNETSKOPE)@\spxentry{main()}\spxextra{in module Project.Script.RecupKbNETSKOPE}}

\begin{fulllineitems}
\phantomsection\label{\detokenize{Project.Script:Project.Script.RecupKbNETSKOPE.main}}
\pysigstartsignatures
\pysiglinewithargsret{\sphinxcode{\sphinxupquote{Project.Script.RecupKbNETSKOPE.}}\sphinxbfcode{\sphinxupquote{main}}}{}{}
\pysigstopsignatures
\sphinxAtStartPar
Execute the entire process of saving the main page’s content to a local file, generating all the pages URLs, and scraping all pages’ content.

\sphinxAtStartPar
This function performs the following steps:
1. Saves the HTML content of the main page to a file named ‘Temp/knowledge.html’.
2. Generates all the pages URLs using the ‘generateAllPagesUrls’ function.
3. Scrapes all pages’ content using the ‘getContent’ function.
4. Removes all temporary files.

\sphinxAtStartPar
This function does not take any parameters and does not return any values.

\end{fulllineitems}

\index{save\_page\_content() (in module Project.Script.RecupKbNETSKOPE)@\spxentry{save\_page\_content()}\spxextra{in module Project.Script.RecupKbNETSKOPE}}

\begin{fulllineitems}
\phantomsection\label{\detokenize{Project.Script:Project.Script.RecupKbNETSKOPE.save_page_content}}
\pysigstartsignatures
\pysiglinewithargsret{\sphinxcode{\sphinxupquote{Project.Script.RecupKbNETSKOPE.}}\sphinxbfcode{\sphinxupquote{save\_page\_content}}}{\sphinxparam{\DUrole{n}{url}}}{}
\pysigstopsignatures
\sphinxAtStartPar
Save the content of a web page to a file in the local file system.
\begin{description}
\sphinxlineitem{Args:}
\sphinxAtStartPar
url (str): The URL of the web page.

\sphinxlineitem{Returns:}
\sphinxAtStartPar
None

\end{description}

\end{fulllineitems}



\paragraph{Project.Script.RecupKbPaloAlto module}
\label{\detokenize{Project.Script:module-Project.Script.RecupKbPaloAlto}}\label{\detokenize{Project.Script:project-script-recupkbpaloalto-module}}\index{module@\spxentry{module}!Project.Script.RecupKbPaloAlto@\spxentry{Project.Script.RecupKbPaloAlto}}\index{Project.Script.RecupKbPaloAlto@\spxentry{Project.Script.RecupKbPaloAlto}!module@\spxentry{module}}
\sphinxAtStartPar
This module contains functions for retrieving the content of Palo Alto’s knowledge base and saving it to a file.
\index{get\_urls() (in module Project.Script.RecupKbPaloAlto)@\spxentry{get\_urls()}\spxextra{in module Project.Script.RecupKbPaloAlto}}

\begin{fulllineitems}
\phantomsection\label{\detokenize{Project.Script:Project.Script.RecupKbPaloAlto.get_urls}}
\pysigstartsignatures
\pysiglinewithargsret{\sphinxcode{\sphinxupquote{Project.Script.RecupKbPaloAlto.}}\sphinxbfcode{\sphinxupquote{get\_urls}}}{\sphinxparam{\DUrole{n}{file\_name}}}{}
\pysigstopsignatures
\sphinxAtStartPar
Read an HTML file and extracts all URLs from anchor tags.
\begin{description}
\sphinxlineitem{Args:}
\sphinxAtStartPar
file\_name (str): The path to the HTML file.

\sphinxlineitem{Returns:}\begin{description}
\sphinxlineitem{List{[}str{]}: A list of URLs extracted from the HTML file.}
\sphinxAtStartPar
Each URL is followed by a newline character.

\end{description}

\end{description}

\end{fulllineitems}

\index{get\_urls\_from\_page\_source() (in module Project.Script.RecupKbPaloAlto)@\spxentry{get\_urls\_from\_page\_source()}\spxextra{in module Project.Script.RecupKbPaloAlto}}

\begin{fulllineitems}
\phantomsection\label{\detokenize{Project.Script:Project.Script.RecupKbPaloAlto.get_urls_from_page_source}}
\pysigstartsignatures
\pysiglinewithargsret{\sphinxcode{\sphinxupquote{Project.Script.RecupKbPaloAlto.}}\sphinxbfcode{\sphinxupquote{get\_urls\_from\_page\_source}}}{\sphinxparam{\DUrole{n}{file\_name}}}{}
\pysigstopsignatures
\sphinxAtStartPar
Read an HTML file and extracts all URLs from anchor tags.
\begin{description}
\sphinxlineitem{Args:}
\sphinxAtStartPar
file\_name (str): The path to the HTML file.

\sphinxlineitem{Returns:}\begin{description}
\sphinxlineitem{List{[}str{]}: A list of URLs extracted from the HTML file.}
\sphinxAtStartPar
Each URL is followed by a newline character.

\end{description}

\end{description}

\end{fulllineitems}

\index{main() (in module Project.Script.RecupKbPaloAlto)@\spxentry{main()}\spxextra{in module Project.Script.RecupKbPaloAlto}}

\begin{fulllineitems}
\phantomsection\label{\detokenize{Project.Script:Project.Script.RecupKbPaloAlto.main}}
\pysigstartsignatures
\pysiglinewithargsret{\sphinxcode{\sphinxupquote{Project.Script.RecupKbPaloAlto.}}\sphinxbfcode{\sphinxupquote{main}}}{}{}
\pysigstopsignatures
\sphinxAtStartPar
Execute the entire process of scraping the Palo Alto knowledge base website.

\sphinxAtStartPar
This function performs the following steps:
1. Saves the HTML content of the main page to a file named ‘Temp/knowledge\_base.html’.
2. Extracts the URLs of the individual pages using regular expressions.
3. Generates additional URLs for each page number from 2 to the maximum number of pages.
4. Saves the URLs to a file named ‘Temp/FirstURLs.txt’.
5. Calls the ‘scrapPages’ function to scrape the pages and their URLs.
6. Removes all temporary files.

\sphinxAtStartPar
This function does not take any parameters and does not return any values.

\end{fulllineitems}

\index{save\_knowledge\_base() (in module Project.Script.RecupKbPaloAlto)@\spxentry{save\_knowledge\_base()}\spxextra{in module Project.Script.RecupKbPaloAlto}}

\begin{fulllineitems}
\phantomsection\label{\detokenize{Project.Script:Project.Script.RecupKbPaloAlto.save_knowledge_base}}
\pysigstartsignatures
\pysiglinewithargsret{\sphinxcode{\sphinxupquote{Project.Script.RecupKbPaloAlto.}}\sphinxbfcode{\sphinxupquote{save\_knowledge\_base}}}{\sphinxparam{\DUrole{n}{url}\DUrole{p}{:}\DUrole{w}{ }\DUrole{n}{str}}\sphinxparamcomma \sphinxparam{\DUrole{n}{file\_name}\DUrole{p}{:}\DUrole{w}{ }\DUrole{n}{str}}}{{ $\rightarrow$ None}}
\pysigstopsignatures
\sphinxAtStartPar
Save the content of a webpage to a file.
\begin{description}
\sphinxlineitem{Args:}
\sphinxAtStartPar
url (str): The URL of the webpage to save.
file\_name (str): The name of the file to save the content to.

\sphinxlineitem{Returns:}
\sphinxAtStartPar
None

\sphinxlineitem{Raises:}
\sphinxAtStartPar
None

\end{description}

\sphinxAtStartPar
This function sends a GET request to the specified URL and retrieves the HTML content of the webpage.
It then uses BeautifulSoup to parse the HTML and prettify it. The prettified HTML is then written to the specified file.

\end{fulllineitems}

\index{save\_page\_source() (in module Project.Script.RecupKbPaloAlto)@\spxentry{save\_page\_source()}\spxextra{in module Project.Script.RecupKbPaloAlto}}

\begin{fulllineitems}
\phantomsection\label{\detokenize{Project.Script:Project.Script.RecupKbPaloAlto.save_page_source}}
\pysigstartsignatures
\pysiglinewithargsret{\sphinxcode{\sphinxupquote{Project.Script.RecupKbPaloAlto.}}\sphinxbfcode{\sphinxupquote{save\_page\_source}}}{\sphinxparam{\DUrole{n}{url}}\sphinxparamcomma \sphinxparam{\DUrole{n}{firstFolderName}}\sphinxparamcomma \sphinxparam{\DUrole{n}{folder}}\sphinxparamcomma \sphinxparam{\DUrole{n}{file\_name}}}{}
\pysigstopsignatures
\sphinxAtStartPar
Save the page source of a given URL to a file.
\begin{description}
\sphinxlineitem{Args:}
\sphinxAtStartPar
url (str): The URL of the page.
firstFolderName (str): The name of the top\sphinxhyphen{}level folder.
folder (str): The name of the subfolder.
file\_name (str): The name of the file to save the page source to.

\sphinxlineitem{Returns:}
\sphinxAtStartPar
None

\end{description}

\end{fulllineitems}

\index{save\_urls() (in module Project.Script.RecupKbPaloAlto)@\spxentry{save\_urls()}\spxextra{in module Project.Script.RecupKbPaloAlto}}

\begin{fulllineitems}
\phantomsection\label{\detokenize{Project.Script:Project.Script.RecupKbPaloAlto.save_urls}}
\pysigstartsignatures
\pysiglinewithargsret{\sphinxcode{\sphinxupquote{Project.Script.RecupKbPaloAlto.}}\sphinxbfcode{\sphinxupquote{save\_urls}}}{\sphinxparam{\DUrole{n}{urls}\DUrole{p}{:}\DUrole{w}{ }\DUrole{n}{List\DUrole{p}{{[}}str\DUrole{p}{{]}}}}\sphinxparamcomma \sphinxparam{\DUrole{n}{file\_name}\DUrole{p}{:}\DUrole{w}{ }\DUrole{n}{str}}}{{ $\rightarrow$ None}}
\pysigstopsignatures
\sphinxAtStartPar
Save a list of URLs to a file.
\begin{description}
\sphinxlineitem{Args:}
\sphinxAtStartPar
urls (List{[}str{]}): The list of URLs to save.
file\_name (str): The name of the file to save the URLs to.

\sphinxlineitem{Returns:}
\sphinxAtStartPar
None

\end{description}

\end{fulllineitems}

\index{scrapPages() (in module Project.Script.RecupKbPaloAlto)@\spxentry{scrapPages()}\spxextra{in module Project.Script.RecupKbPaloAlto}}

\begin{fulllineitems}
\phantomsection\label{\detokenize{Project.Script:Project.Script.RecupKbPaloAlto.scrapPages}}
\pysigstartsignatures
\pysiglinewithargsret{\sphinxcode{\sphinxupquote{Project.Script.RecupKbPaloAlto.}}\sphinxbfcode{\sphinxupquote{scrapPages}}}{\sphinxparam{\DUrole{n}{FirstUrls}}}{}
\pysigstopsignatures
\sphinxAtStartPar
Scrap pages based on the list of URLs provided in FirstUrls.

\sphinxAtStartPar
Check page source for search results and append URLs accordingly.
Save page sources and extract URLs for further processing.
\begin{description}
\sphinxlineitem{Args:}
\sphinxAtStartPar
FirstUrls (list): List of URLs to scrape pages from

\sphinxlineitem{Returns:}
\sphinxAtStartPar
None

\end{description}

\end{fulllineitems}

\index{selenium() (in module Project.Script.RecupKbPaloAlto)@\spxentry{selenium()}\spxextra{in module Project.Script.RecupKbPaloAlto}}

\begin{fulllineitems}
\phantomsection\label{\detokenize{Project.Script:Project.Script.RecupKbPaloAlto.selenium}}
\pysigstartsignatures
\pysiglinewithargsret{\sphinxcode{\sphinxupquote{Project.Script.RecupKbPaloAlto.}}\sphinxbfcode{\sphinxupquote{selenium}}}{\sphinxparam{\DUrole{n}{url}}}{}
\pysigstopsignatures
\sphinxAtStartPar
Use Selenium to open a URL in a headless browser.
\begin{description}
\sphinxlineitem{Args:}
\sphinxAtStartPar
url (str): The URL to open.

\sphinxlineitem{Returns:}
\sphinxAtStartPar
str: The page source of the opened URL.

\end{description}

\end{fulllineitems}



\paragraph{Project.Script.RecupKbZscaler module}
\label{\detokenize{Project.Script:module-Project.Script.RecupKbZscaler}}\label{\detokenize{Project.Script:project-script-recupkbzscaler-module}}\index{module@\spxentry{module}!Project.Script.RecupKbZscaler@\spxentry{Project.Script.RecupKbZscaler}}\index{Project.Script.RecupKbZscaler@\spxentry{Project.Script.RecupKbZscaler}!module@\spxentry{module}}
\sphinxAtStartPar
This module retrieves the content of Zscaler’s knowledge base and saves it to a file.
\index{getAllCategory() (in module Project.Script.RecupKbZscaler)@\spxentry{getAllCategory()}\spxextra{in module Project.Script.RecupKbZscaler}}

\begin{fulllineitems}
\phantomsection\label{\detokenize{Project.Script:Project.Script.RecupKbZscaler.getAllCategory}}
\pysigstartsignatures
\pysiglinewithargsret{\sphinxcode{\sphinxupquote{Project.Script.RecupKbZscaler.}}\sphinxbfcode{\sphinxupquote{getAllCategory}}}{\sphinxparam{\DUrole{n}{file}}}{}
\pysigstopsignatures
\sphinxAtStartPar
Read a file, parses its content using BeautifulSoup, extracts all the links from the parsed content, and filters them to get single segment links. It then writes the filtered links to a new file, removes irrelevant links and writes the remaining links to a new file.
\begin{description}
\sphinxlineitem{Args:}
\sphinxAtStartPar
file (str): The path to the file to be read.

\sphinxlineitem{Returns:}
\sphinxAtStartPar
None

\end{description}

\end{fulllineitems}

\index{getAllPagesUrls() (in module Project.Script.RecupKbZscaler)@\spxentry{getAllPagesUrls()}\spxextra{in module Project.Script.RecupKbZscaler}}

\begin{fulllineitems}
\phantomsection\label{\detokenize{Project.Script:Project.Script.RecupKbZscaler.getAllPagesUrls}}
\pysigstartsignatures
\pysiglinewithargsret{\sphinxcode{\sphinxupquote{Project.Script.RecupKbZscaler.}}\sphinxbfcode{\sphinxupquote{getAllPagesUrls}}}{\sphinxparam{\DUrole{n}{file}}}{}
\pysigstopsignatures
\sphinxAtStartPar
Read a file with URLs, retrieves the source code of each URL, extracts links from the source code, and writes the extracted links to a new file.
\begin{description}
\sphinxlineitem{Args:}
\sphinxAtStartPar
file (str): The path to the file containing URLs.

\sphinxlineitem{Returns:}
\sphinxAtStartPar
None

\end{description}

\end{fulllineitems}

\index{getContent() (in module Project.Script.RecupKbZscaler)@\spxentry{getContent()}\spxextra{in module Project.Script.RecupKbZscaler}}

\begin{fulllineitems}
\phantomsection\label{\detokenize{Project.Script:Project.Script.RecupKbZscaler.getContent}}
\pysigstartsignatures
\pysiglinewithargsret{\sphinxcode{\sphinxupquote{Project.Script.RecupKbZscaler.}}\sphinxbfcode{\sphinxupquote{getContent}}}{\sphinxparam{\DUrole{n}{file}}}{}
\pysigstopsignatures
\sphinxAtStartPar
Read URLs from a file, save their content to the local file system.
\begin{description}
\sphinxlineitem{Args:}
\sphinxAtStartPar
file (str): The path to the file containing the URLs.

\sphinxlineitem{Returns:}
\sphinxAtStartPar
None

\end{description}

\end{fulllineitems}

\index{getSourceCode() (in module Project.Script.RecupKbZscaler)@\spxentry{getSourceCode()}\spxextra{in module Project.Script.RecupKbZscaler}}

\begin{fulllineitems}
\phantomsection\label{\detokenize{Project.Script:Project.Script.RecupKbZscaler.getSourceCode}}
\pysigstartsignatures
\pysiglinewithargsret{\sphinxcode{\sphinxupquote{Project.Script.RecupKbZscaler.}}\sphinxbfcode{\sphinxupquote{getSourceCode}}}{\sphinxparam{\DUrole{n}{url}\DUrole{p}{:}\DUrole{w}{ }\DUrole{n}{str}}\sphinxparamcomma \sphinxparam{\DUrole{n}{file}\DUrole{p}{:}\DUrole{w}{ }\DUrole{n}{str}}}{{ $\rightarrow$ None}}
\pysigstopsignatures
\sphinxAtStartPar
Retrieve the source code of a webpage and saves it to a file.
\begin{description}
\sphinxlineitem{Args:}
\sphinxAtStartPar
url (str): The URL of the webpage to retrieve the source code from.
file (str): The path to the file where the source code will be saved.

\sphinxlineitem{Returns:}
\sphinxAtStartPar
None

\end{description}

\end{fulllineitems}

\index{main() (in module Project.Script.RecupKbZscaler)@\spxentry{main()}\spxextra{in module Project.Script.RecupKbZscaler}}

\begin{fulllineitems}
\phantomsection\label{\detokenize{Project.Script:Project.Script.RecupKbZscaler.main}}
\pysigstartsignatures
\pysiglinewithargsret{\sphinxcode{\sphinxupquote{Project.Script.RecupKbZscaler.}}\sphinxbfcode{\sphinxupquote{main}}}{}{}
\pysigstopsignatures
\sphinxAtStartPar
Execute the entire process of scraping the Zscaler knowledge base website.

\sphinxAtStartPar
This function performs the following steps:
1. Saves the HTML content of the main page to a file.
2. Extracts all the categories from the parsed content.
3. Generates all the pages URLs for each category.
4. Scrapes all pages’ content for each category.
5. Removes all temporary files.

\sphinxAtStartPar
This function does not take any parameters and does not return any values.

\end{fulllineitems}

\index{save\_page\_content() (in module Project.Script.RecupKbZscaler)@\spxentry{save\_page\_content()}\spxextra{in module Project.Script.RecupKbZscaler}}

\begin{fulllineitems}
\phantomsection\label{\detokenize{Project.Script:Project.Script.RecupKbZscaler.save_page_content}}
\pysigstartsignatures
\pysiglinewithargsret{\sphinxcode{\sphinxupquote{Project.Script.RecupKbZscaler.}}\sphinxbfcode{\sphinxupquote{save\_page\_content}}}{\sphinxparam{\DUrole{n}{url}}}{}
\pysigstopsignatures
\sphinxAtStartPar
Save the content of a web page to a file in the local file system.
\begin{description}
\sphinxlineitem{Args:}
\sphinxAtStartPar
url (str): The URL of the web page.

\sphinxlineitem{Returns:}
\sphinxAtStartPar
None

\end{description}

\end{fulllineitems}



\paragraph{Module contents}
\label{\detokenize{Project.Script:module-Project.Script}}\label{\detokenize{Project.Script:module-contents}}\index{module@\spxentry{module}!Project.Script@\spxentry{Project.Script}}\index{Project.Script@\spxentry{Project.Script}!module@\spxentry{module}}

\subsection{Submodules}
\label{\detokenize{Project:submodules}}

\subsection{Project.run module}
\label{\detokenize{Project:module-Project.run}}\label{\detokenize{Project:project-run-module}}\index{module@\spxentry{module}!Project.run@\spxentry{Project.run}}\index{Project.run@\spxentry{Project.run}!module@\spxentry{module}}
\sphinxAtStartPar
This module contains the main function that orchestrates the execution of scripts to scrape knowledge bases for Fortinet, Palo Alto, NetsKope, and Zscaler.
\index{main() (in module Project.run)@\spxentry{main()}\spxextra{in module Project.run}}

\begin{fulllineitems}
\phantomsection\label{\detokenize{Project:Project.run.main}}
\pysigstartsignatures
\pysiglinewithargsret{\sphinxcode{\sphinxupquote{Project.run.}}\sphinxbfcode{\sphinxupquote{main}}}{}{}
\pysigstopsignatures
\sphinxAtStartPar
Orchestrate the execution of scripts to scrape knowledge bases for Fortinet, Palo Alto, NetsKope, and Zscaler.

\end{fulllineitems}



\subsection{Module contents}
\label{\detokenize{Project:module-Project}}\label{\detokenize{Project:module-contents}}\index{module@\spxentry{module}!Project@\spxentry{Project}}\index{Project@\spxentry{Project}!module@\spxentry{module}}

\chapter{Indices and tables}
\label{\detokenize{index:indices-and-tables}}\begin{itemize}
\item {} 
\sphinxAtStartPar
\DUrole{xref,std,std-ref}{genindex}

\item {} 
\sphinxAtStartPar
\DUrole{xref,std,std-ref}{modindex}

\item {} 
\sphinxAtStartPar
\DUrole{xref,std,std-ref}{search}

\end{itemize}


\renewcommand{\indexname}{Python Module Index}
\begin{sphinxtheindex}
\let\bigletter\sphinxstyleindexlettergroup
\bigletter{p}
\item\relax\sphinxstyleindexentry{Project}\sphinxstyleindexpageref{Project:\detokenize{module-Project}}
\item\relax\sphinxstyleindexentry{Project.run}\sphinxstyleindexpageref{Project:\detokenize{module-Project.run}}
\item\relax\sphinxstyleindexentry{Project.Script}\sphinxstyleindexpageref{Project.Script:\detokenize{module-Project.Script}}
\item\relax\sphinxstyleindexentry{Project.Script.RecupKbFortinet}\sphinxstyleindexpageref{Project.Script:\detokenize{module-Project.Script.RecupKbFortinet}}
\item\relax\sphinxstyleindexentry{Project.Script.RecupKbNETSKOPE}\sphinxstyleindexpageref{Project.Script:\detokenize{module-Project.Script.RecupKbNETSKOPE}}
\item\relax\sphinxstyleindexentry{Project.Script.RecupKbPaloAlto}\sphinxstyleindexpageref{Project.Script:\detokenize{module-Project.Script.RecupKbPaloAlto}}
\item\relax\sphinxstyleindexentry{Project.Script.RecupKbZscaler}\sphinxstyleindexpageref{Project.Script:\detokenize{module-Project.Script.RecupKbZscaler}}
\end{sphinxtheindex}

\renewcommand{\indexname}{Index}
\printindex
\end{document}